\documentclass{beamer}
\usetheme{Antibes}
\usepackage{graphicx}
\graphicspath{ {/home/kyle/github/IF-neuronal-network/tdmi/figure-eps/Apr24/} }
\usepackage{enumerate}
\usepackage{multirow}
\newcommand{\tabincell}[2]{\begin{tabular}{@{}#1@{}}#2\end{tabular}}
\usepackage{subfigure}

\title{Report of I\&F Neuronal Network Simulation}
\subtitle{Report of Week 10}
\author{Kai Chen}
\date{Apr. 25, 2017}

\begin{document}
	% title frame
	\frame{\titlepage}
	% frame 2
	\section{Key Questions}
	\begin{frame}
		\frametitle{Contents}
		\begin{enumerate}[*]
			\item Intensity of TDMI signals influenced by post network neuronal number contributed to LFP as well as their dynamical state.
			\item Comparison between TDMI signal of spike trains of excitatory pre-network neurons and their one-degree LFP and inhibitory neuronal spike trains and theirs. 
			\item Contribution of excitatory and inhibitory neuron to LFP in post-network based on point current source model of local field potential.
			\item Study of two-degree LFPs and their TDMI signals.
		\end{enumerate}
	\end{frame}
	\section{Simulation Setups}
	\subsection{Network Structures}
	\begin{frame}
		\frametitle{General simulation setups}
		Basically, the simulating system consists of two one dimensional integrate-and-fire neuronal network, called it two-network system. 
		\begin{enumerate}
			\scriptsize
			\item Each network contains 100 neurons;
			\item Neuronal connecting density is 3, which means that each neuron directly connects with 6 other neurons within its own loop;
			\item Each network has a regular structure, which means each of its neuron connects with 6 nearest neurons;
			\item Each neuron in loop 1 has 10\% probability connecting with every single neuron in loop 2;
			\item Spiking signals generatd by one of those two network, which called 'loop 1' or pre-network, can transmit to the other network, which I called it 'loop 2' or post-network; and the spiking information flow cannot transmit backwards;
			\item simulation time range is [0, 10] s; timing step = 1/32 ms
		\end{enumerate}		
	\end{frame}
	\subsection{Neuronal Parameters}
	\begin{frame}
		\frametitle{Neuronal Parameters}
		\begin{enumerate}
			\item Neurons are driven by feedforward excitatory spiking signal generated by Poisson process with 1.5 \emph{kHz} mean driving rate; Roughly speaking, a neuron receives 30 spikes from feedforward Poisson process and its membrane potential, as a result, would increase from resting potential to firing threshold.
			\item The intensity of neuron-neuron interaction is identical to the intensity of feedforward inputs; And the effect of single neuron towards the change of conductance of ion channel for both excitatory and inhibitory are identical, too.
		\end{enumerate}
	\end{frame}
	
	\subsection{About LFP model}
	\begin{frame}
	\frametitle{About LFP model}
		\begin{enumerate}[*]
			\item In the following results, I choose neuronal data, including membrane potential and different component of membrane conductance, from 1000 to 10000 ms to generate local field potential, which can prevent the disturbance from the transition process of network dynamics at the first one second.
			\item One-degree LFP: local field potential generated by neurons directly connected with target neuron in post-network;
			\item Two-degree LFP: local field potential generated by neurons connected with first order connected neurons of target neuron; a neuron cannot be first order neuron and second order neuron at the same time;
		\end{enumerate}
	\end{frame}

	\section{Intensity of TDMI signals influenced by post network neuronal number contributed to LFP as well as their dynamical state.}
	\subsection{General Views}
	\begin{frame}
	\frametitle{General Views}
		\scriptsize{All neurons in the system are identical excitatory neurons.} 
		\begin{figure}
			\centering
			\includegraphics[height = 5cm, width = 10cm]{snr-1-50-1000_10000-0_25-60-100.eps}
			\caption{\scriptsize{signal-noise ratio is defined as the ratio of the maximum value of effective MI signal and mean value of noise level of TDMI. Here, the results reflects that all neuron in pre-network have effective MI signals with their one-degree LFPs.}}
			\label{fig:snr_all_1}
		\end{figure}
	\end{frame}

	\subsection{Sub-clustering cases}
	\begin{frame}
	\frametitle{One neuron case}
		Choose \#0 neuron in loop-1 as the target neurons for spike train. It randomly connects with 9 neurons in loop-2. Now, we regard the membrane current of each neuron as LFP, and apply TDMI analysis among them. Results are shown below.
		\begin{table}[h]
		\scriptsize
		\begin{tabular}{l l l l l}
			\hline
			index & mean firing rate & signal noise ratio & peak time & time constant \\
			\hline
			5  & 22.30 & 2.840166 & 0.50 & 0.776910 \\
			32 & 21.60 & 2.214518 & -1   & -1 \\
			37 & 20.40 & 2.638657 & 0.25 & 0 \\
			43 & 21.70 & 2.452756 & 0.25 & 1.510032 \\
			45 & 22.60 & 2.007208 & -1	 & -1 \\
			52 & 26.50 & 2.009227 & -1   & -1 \\
			57 & 22.00 & 2.252543 & 0.25 & 0 \\
			83 & 21.10 & 2.735426 & -1   & -1 \\
			84 & 21.10 & 2.906195 & 0.25 & 0 \\
			\hline
		\end{tabular}
		\end{table}
		\footnotesize{\emph{Denote: If peak time equals to -1, there is no MI signal in TDMI graph. If peak time is positive and time constant equals to 0, there is a weak MI signal while it cannot be treated as an effective signal.}}
	\end{frame}

	\begin{frame}
	\frametitle{One neuron case - Comparison}
		\footnotesize{According to the tabular in previous slide, for most of cases, there is no effective TDMI signal. Here, we show the TDMI graphs from \#5, \#43 and \#45 neuron.}
		\begin{figure}[h]
			\centering
			\subfigure[\#5 neuron]{\includegraphics[width = 3.5cm]{/t3-0/tdmi-1/tdmi-5-1-1000_10000-all-50-0_25-60-100.eps}}
			\subfigure[\#43 neuron]{\includegraphics[width = 3.5cm]{/t3-0/tdmi-1/tdmi-43-1-1000_10000-all-50-0_25-60-100.eps}}
			\subfigure[\#45 neuron]{\includegraphics[width = 3.5cm]{/t3-0/tdmi-1/tdmi-45-1-1000_10000-all-50-0_25-60-100.eps}}
			\label{fig:tdmi1}
		\end{figure}
		ANALYSIS: Based on this comparsion, we can observe weak MI signals in \#5 neuron and \#43 neuron cases, while find nothing in \#45 neuron case. Since all these three neurons are connected with identical neuron in loop-1, we make an assumption that such a difference results from different dynamical states that neurons have.
	\end{frame}

	\begin{frame}
	\frametitle{One neuron case - Comparison}
	\begin{figure}[h]
		\centering
		\begin{minipage}[ht]{0.48\linewidth}
			\centering
			\includegraphics[width = 1\linewidth]{corr-1.eps}
			\caption{Cross correlation between total membrane current of neurons.}
		\end{minipage}
		\begin{minipage}[ht]{0.48\linewidth}
			\centering
			\includegraphics[width = 1\linewidth]{corrV-1.eps}
			\caption{Cross correlation between membrane potential of neurons}
		\end{minipage}
		\caption{Cross correlation between single neuronal states}
		\label{fig:corr-1}
	\end{figure}
	\end{frame}
	
	\begin{frame}
	\frametitle{Multi-neuron case}
	\small{For multi-neuron case, here, we take \#0, \#40 and \#80 neurons as examples.
	Brief summary of properties of these neurons are listed below.}
	\begin{table}[h]
	\footnotesize
	\centering
	\begin{tabular}{l l l l}
		\hline
		Index & Type & No. of direct connection & Mean firing rate (Hz)\\
		\hline
		0			& exc  & 9 	& 14.50\\
		40		& exc  & 11 & 13.80\\
		80		& exc  & 5 	& 14.00\\
		\hline
	\end{tabular}
	\end{table}
	\small{We calculate the signal-noise ratio of TDMI signal for all possible sub-clusters in their directly connecting neuronal cluster, and make a few plot to show their dependence towards some parameters.}
	\end{frame}

	\begin{frame}
	\frametitle{Signal-Noise Ratio vs. Neuronal Number in Clusters}
	\begin{figure}[h]
		\centering
		\begin{minipage}[ht]{0.49\linewidth}
			\centering
			\includegraphics[width = 1\linewidth]{/t3-0/snr-boxplot.png}
			\caption{\#0 neuron}
		\end{minipage}
		\begin{minipage}[ht]{0.49\linewidth}
			\centering
			\includegraphics[width = 1\linewidth]{/t3-40/snr-boxplot.png}
			\caption{\#40 neuron}
		\end{minipage}
		\label{fig:snr_box}
	\end{figure}
	\end{frame}

	\begin{frame}
	\frametitle{Signal-Noise Ratio vs. Mean Firing Rate of Clusters}
	\begin{figure}[h]
		\centering
		\begin{minipage}[ht]{0.49\linewidth}
			\centering
			\includegraphics[width = 1\linewidth]{/t3-0/mrate-snr.png}
			\caption{\#0 neuron}
		\end{minipage}
		\begin{minipage}[ht]{0.49\linewidth}
			\centering
			\includegraphics[width = 1\linewidth]{/t3-40/mrate-snr.png}
			\caption{\#40 neuron}
		\end{minipage}
		\label{fig:mrate_snr}
	\end{figure}
	\end{frame}

	\section{Comparison between TDMI signal of spike trains from excitatory neurons and inhibitory ones}
	\subsection{General Views}
	\begin{frame}
	\frametitle{General Views}
		\footnotesize{Neurons in pre-network are 80\% excitatory and 20\% inhibitory. All neurons in post-network are identical excitatory neurons.} 
		\begin{figure}
			\centering
			\includegraphics[height = 5cm, width = 9cm]{snr-all-t7.eps}
			\caption{\footnotesize{Signal-noise ratio of TDMI signal for all neurons with their one-degree LFP.}}
			\label{fig:snr_all_2}
		\end{figure}
	\end{frame}

	\subsection{TDMI signal depends on neuronal type}
	\begin{frame}
	\frametitle{Neuronal Type vs. No.of Connection vs. Signal-noise Ratio}
		\begin{figure}[h]
			\centering
			\includegraphics[width = 8cm]{snr-scatter-t7.png}
			\label{fig:type_snr_all}
		\end{figure}
		\footnotesize{Number of connection is not an essential factor that leads TDMI signal pattern of inhibitory neuron.}
	\end{frame}

	\begin{frame}
	\frametitle{Neuronal Type vs. Mean Firing Rate vs. Signal-noise Ratio}
		\begin{figure}[h]
			\centering
			\includegraphics[width = 8cm]{mrate-snr-all-t7.png}
			\label{fig:mrate_snr_2}
		\end{figure}
		\footnotesize{Mean Firing Rate is not an essential factor that leads to TDMI signal pattern of inhibitory neuron.}
	\end{frame}

	% \section{Averaging theory for TDMI behavior based on weightless point current source LFP model}
	% \begin{frame}
	% \frametitle{Summary}

	% \end{frame}

	\section{Contribution of excitatory and inhibitory neuron to LFP in post-network based on point current source model of local field potential.}
	\subsection{General Views}
	\begin{frame}
	\frametitle{General Views}
		\footnotesize{All neurons in pre-network are identical excitatory neurons. Neurons in post-network are 80\% excitatory and 20\% inhibitory. } 
		\begin{figure}
			\centering
			\includegraphics[height = 5cm, width = 9cm]{snr-all-t6.png}
			\caption{\footnotesize{Signal-noise ratio of TDMI signal for all neurons with their one-degree LFP.}}
			\label{fig:snr_all_2}
		\end{figure}
	\end{frame}

	\begin{frame}
	\frametitle{Signal-noise ratio vs. Neuronal Type}
	\small{\#0 neuron: Connects with 5 excitatory neurons and 4 inhibitory neurons}
		\begin{figure}[h]
			\centering
			\begin{minipage}[ht]{0.49\linewidth}
				\centering
				\includegraphics[width = 1\linewidth]{t6-0/mrate-snr1.png}
				\caption{Colorbar represents number of connections}
			\end{minipage}
			\begin{minipage}[ht]{0.49\linewidth}
				\centering
				\includegraphics[width = 1\linewidth]{t6-0/mrate-snr.png}
				\caption{Colorbar represents portion of excitatory neurons in cluster}
			\end{minipage}
			\label{fig:mrate_snr_t6-0}
		\end{figure}
		\footnotesize{Neuron type in post-network clusters is not an essential factor that leads to effective TDMI signal pattern.}
	\end{frame}

	\begin{frame}
	\frametitle{Signal-noise ratio vs. Neuronal Type}
	\small{\#92 neuron: Connects with 5 excitatory neurons and 5 inhibitory neurons}
		\begin{figure}[h]
			\centering
			\begin{minipage}[ht]{0.49\linewidth}
				\centering
				\includegraphics[width = 1\linewidth]{t6-91/mrate-snr2.png}
				\caption{Colorbar represents number of connections}
			\end{minipage}
			\begin{minipage}[ht]{0.49\linewidth}
				\centering
				\includegraphics[width = 1\linewidth]{t6-91/mrate-snr1.png}
				\caption{Colorbar represents portion of excitatory neurons in cluster}
			\end{minipage}
			\label{fig:mrate_snr_t6-0}
		\end{figure}
		\footnotesize{Neuron type in post-network clusters is not an essential factor that leads to effective TDMI signal pattern.}
	\end{frame}

	\section{Study of two-degree LFPs and their TDMI signals.}
	\begin{frame}
		\frametitle{General Views}
		\footnotesize{All neurons in system are identical excitatory neurons.} 
		\begin{figure}
			\centering
			\includegraphics[height = 5cm, width = 9cm]{snr-all-2-t3.png}
			\caption{\footnotesize{Signal-noise ratio of TDMI signal for all neurons with their two-degree LFP.}}
			\label{fig:snr_all_2}
		\end{figure}
	\end{frame}


	\subsection{excitatory neurons}

	% \begin{frame}
	% 	\frametitle{Figure descriptions}
	% 	\begin{columns}[c]
	% 		\column{.6\textwidth}
						
	% 		\column{.4\textwidth}\scriptsize
	% 		\begin{enumerate}[-]
	% 		\item The title illustrates the index of target neuron, the time range of its data adapted in TDMI analysis, and the detailed parameter applied in TDMI calculation;
	% 		\item The text box in figure shows the propertis of target neuron in loop 1 and the number of neurons contributes to LFP as well as their types in loop 2;
	% 		\item The blue line indicates the TDMI of original spike train and local field potential, and the green line indicates that of randomly swapped spike train and local field potential under the same testing condition;
	% 		\end{enumerate}
	% 	\end{columns}
	% \end{frame}
	
	% \begin{frame}
	% \frametitle{Notations in Figures}
	% \begin{tabular}{l | l} 
	% 	\hline
	% 	Notations & Meaning \\
	% 	\hline
	% 	expected occupancy & \tabincell{l}{expected occupancy of \\ each bin in historgram of LFP} \\
	% 	\hline
	% 	dt & \tabincell{l}{actual time interval for\\ single time-delayed step} \\
	% 	\hline
	% 	NTD & \tabincell{l}{maximum number of negative\\ time-delayed step in the graph} \\
	% 	\hline
	% 	PTD & \tabincell{l}{maximum number of positive\\ time-delayed step} \\
	% 	\hline
	% \end{tabular}
	% \end{frame}
	
	% \begin{frame}
	% \frametitle{TDMI of spike train and LFP} 
	% \begin{columns}[c]
	% \column{.5\textwidth}
	% \begin{figure}
	% \centering
	% \includegraphics[width = 6cm]{tdmi-20-1-1000_10000-all-50-0_25-60-100.eps}	
	% \label{fig:2}
	% \end{figure}		
	% \column{.5\textwidth}
	% \begin{figure}
	% \centering
	% \includegraphics[width = 6cm]{tdmi-47-1-1000_10000-all-50-0_25-60-100.eps}	
	% \label{fig:4}
	% \end{figure}
	% \end{columns}
	% \end{frame}
	
	% \begin{frame}
	% \frametitle{Dynamics of pre-synaptic neurons}
	% In the simulating system, \#20 neuron is excitatory and \#47 neuron is inhibitory. Under the driving of feedforward inputs with same average rate, which is 1500 Hz, their average firing rate from 1s to 10s is 14 Hz and 12 Hz respectively. However, comparing the TDMI signal which they interact with their directly connected neuron clusters, TDMI in excitatory cases do indicate the neuronal interactions and the direction of information flow, while in inhibitory cases it fails.
	% \end{frame}
	
	% \begin{frame}
	% \frametitle{Conclusion 1}
	% \textbf{The correlation between excitatory neuron and its first order connected neurons can be clearly reflected by TDMI plot, while those of inhibitory neuron cannot.}\par
	% \emph{Denote: the conclusion is drawed after 20 neurons in loop 1 were investigate. Among those target neurons, 13 of them are excitatory, others are inhibitory.}
	% \end{frame}
	% \section{The magnitude of MI peak}
	% \subsection{TDMI of \# 20 neuron and LFP from different portion of first order connected neurons}
	
	% \begin{frame}
	% \frametitle{LFP contributed by excitatory neurons}
	% \begin{figure}
	% 	\centering
	% 	\includegraphics[width=10cm]{comp-tdmi-20/comp-tdmi-20.eps}
	% \end{figure}
	% \end{frame}
		
	% \begin{frame}
	% \frametitle{Figure descriptions}
	% Here, I plot TDMI signal of \#20 neuron's spike train and several sub set of its first order connected neuronal clusters. The index contained in the legend "tdmi-i" represents that the sub neuronal cluster consists of i neurons which are random chosen from all first order connected neurons. The legend "tdmi-rand" represents the TDMI of randomly swapped spike train and LFP, which reflects the noise level of MI signal.\par
	% \end{frame}
	
	% \begin{frame}
	% \frametitle{Conclusion 2}
	% The magitude of MI signal depends on the number of neurons in neuron cluster of LFP. Here is a hypothesis. Since each neuron in the cluster of LFP contains information from pre-network neuron and its neighborhood in post-network. Neurons have more connections with their neighbors rather than cross network connections. When the number of LFP cluster is small, the current signal of neuron is dominant by local network dynamics which is not correlated across neurons in cluster. When the number goes up, those non-correlated signal is averaged. However, their cross-network signals originate from the same neuron in pre-network. The effect of these parts would accumulate, and as a result, MI signal would increase.
	
	% \end{frame}
	% \section{The contribution of excitatory and inhibitory neurons to LFP}
	% \begin{frame}
	% \frametitle{For excitatory neurons:}
	% All excitatory neurons in first-order connected neuron cluster are chosen as a sub-set to generate another LFP sequence. So do inhibitory neurons.
	% \begin{columns}[b]
	% \column{.5\textwidth}
	% \begin{figure}
	% \centering
	% \includegraphics[height = 5cm]{tdmi-79-1-1000_10000-exc-50-0_25-60-100.eps}	
	% \label{fig:8}
	% \end{figure}
	% \column{.5\textwidth}
	% \begin{figure}
	% \includegraphics[height = 5cm]{tdmi-79-1-1000_10000-inh-50-0_25-60-100.eps}	
	% \label{fig:9}
	% \end{figure}
	% \end{columns}	
	% \end{frame}
	
	% \begin{frame}
	% \frametitle{For excitatory neurons:}
	% \begin{columns}[b]
	% \column{.5\textwidth}
	% \begin{figure}
	% \centering
	% \includegraphics[height = 5cm]{tdmi-92-1-1000_10000-exc-50-0_25-60-100.eps}	
	% \label{fig:8}
	% \end{figure}
	% \column{.5\textwidth}
	% \begin{figure}
	% \includegraphics[height = 5cm]{tdmi-92-1-1000_10000-inh-50-0_25-60-100.eps}	
	% \label{fig:9}
	% \end{figure}
	% \end{columns}	
	% \end{frame}
	
	% \begin{frame}
	% \frametitle{For inhibitory neurons:}
	% \begin{columns}[b]
	% \column{.5\textwidth}
	% \begin{figure}
	% \centering
	% \includegraphics[height = 5cm]{tdmi-87-1-1000_10000-exc-50-0_25-60-100.eps}	
	% \label{fig:10}
	% \end{figure}
	
	% \column{.5\textwidth}
	% \begin{figure}
	% \includegraphics[height = 5cm]{tdmi-87-1-1000_10000-inh-50-0_25-60-100.eps}	
	% \label{fig:11}
	% \end{figure}
	% \end{columns}
	% \end{frame}
	
	% \begin{frame}
	% \frametitle{Conclusion 3}
	% Excitatory and inhibitory neurons in post-network contribute the similar effect on the amount of MI.
	% \end{frame}
	
	% \section{Effects from Second order connected neuron on MI signal.}
	% \begin{frame}
	% \frametitle{TDMI of spike train and $2^{nd}$ order LFP}
	
	% \begin{columns}[b]
	% \column{.5\textwidth}
	% \begin{figure}
	% \centering
	% \includegraphics[height = 5cm]{tdmi-20-2-1000_10000-all-50-0_25-60-100.eps}	
	% \label{fig:33}
	% \end{figure}
		
	% \column{.5\textwidth}
	% \begin{figure}
	% \centering
	% \includegraphics[height = 5cm]{tdmi-47-2-1000_10000-all-50-0_25-60-100.eps}	
	% \label{fig:33}
	% \end{figure}
	% \end{columns}
	% \end{frame}
	
	% \subsection{Conclusion}
	% \begin{frame}
	% \frametitle{Conclusion 4}
	% \textbf{The TDMI of spike trains for both excitatory and inhibitory neuron and second order LFP fluctuate at noise level. As a result, we can see the $2^{nd}$ order connecting pattern between neurons in terms of TDMI plot.}\par
	% \emph{Denote: This conclusion is drawn in current neuron-neuron interaction scheme. If the intensity of neuronal interaction increases, there might be different results. This case would be discussed in my next report.}
	% \end{frame}
	% \section{Working plans}
	% \begin{frame}
	% \frametitle{Working plans for next week}
	% \begin{enumerate}
	% 	\item Check the dynamical states of different neuron in both pre- and post- networks. Study the different impacts toward MI from cross-network communication and neuronal interaction within single network.
	% 	\item Study the time delay between MI peak and zero time point; Find the parameter that impact the length of it by changing parameters related to neuron-neuron interaction;
	% 	\item Try to run test on networks with different network structures, to see whether there is any different TDMI pattern;
	% \end{enumerate}
	% \end{frame}

\end{document}

