\documentclass{beamer}
\usetheme{Antibes}
\usepackage{graphicx}
\graphicspath{ {/home/kyle/github/IF-neuronal-network/tdmi/figure-eps/} }
\usepackage{enumerate}
\usepackage{subfigure}
\usepackage{multirow}
% \setbeamertemplate{navigation symbols}{}

\title{Report for I\&F Neuronal Network Simulation}
\subtitle{Week 11}
\author{Kai Chen}
\date{May. 5, 2017}

\begin{document}
\frame{\titlepage}

\section{Network Dynamics based on High-rate burst firing interneurons}

\begin{frame}
  \frametitle{Raster Plots}
  \tiny{System setups: Pre-network consists 74 excitatory neurons and 26 inhibitory neurons. And post-network contains 100 excitatory neurons. For excitatory neurons, they are driven by 1.5 kHz feedforward Poisson process, while inhibitory ones are driven by 4.5 kHz Poisson process.}
  \begin{figure}[h]
    \centering
    \subfigure{\includegraphics[width = 8cm, height = 2.5cm]{May03/t2/raster-pre.png}}
    \subfigure{\includegraphics[width = 8cm, height = 2cm]{May03/t2/hist-pre.png}}
  \end{figure}
  \tiny{Raster plot and histogram of spikes. In the lower panel, vertical axis indicates the number of spikes per milisecond.}
\end{frame}

\begin{frame}
  \frametitle{Mean Firing Rate}
  \tiny{Pink dots represent that spike train in TMDI is from excitatory pre-network neurons, and blue ones represents those from inhibitory neurons.}
  \begin{figure}[h]
    \centering
    \includegraphics[width = 0.78\linewidth]{May03/t2/rate-ind.png}
  \end{figure}
\end{frame}

\begin{frame}
  \frametitle{Signal Noise Ratio}
  \tiny{Pink dots represent that spike train in TMDI is from excitatory pre-network neurons, and blue ones represents those from inhibitory neurons.}
  \begin{figure}[h]
    \centering
    \includegraphics[width = 0.78\linewidth]{May03/t5/snr-ind.png}
  \end{figure}
  \tiny{Blue dots are located between 2 amd 6 SNR(signal-noise ratio), which indicates the effective TDMI signal of inhibitory spike trains and LFP in this configuration.}
\end{frame}

\section{Comparison between TDMI and TDLC}

\begin{frame}
  \frametitle{Comparison between TDMI and TDLC}
  \scriptsize{Three figure below illustrate the TDMI and TDLC of an excitatory neuron and LFP generated by its 11 directly connected post-network excitatory neurons. The timing step of time delay is 0.25 ms, 0.0625 ms and 0.03125 ms, respectively.}
  \begin{figure}[h]
    \centering
    \subfigure{\includegraphics[width = 0.32\linewidth]{May03/tdmi-tdlc-40-11-025.png}}
    \subfigure{\includegraphics[width = 0.32\linewidth]{May03/tdmi-tdlc-40-11-00625.png}}
    \subfigure{\includegraphics[width = 0.32\linewidth]{May03/tdmi-tdlc-40-11-003125.png}}
  \end{figure}
  \small{By changing timing step in TDLC(time-delayed linear correlation) calculation, program shows completely different results, which is quite an abnormal things. The code for TDLC calculation would be checked again later.}
\end{frame}

\begin{frame}
  \frametitle{Comparison between TDMI and TDLC}
  \scriptsize{Three figure below illustrate the TDMI and TDLC of an excitatory neuron and LFP generated by its directly connected post-network excitatory neurons. The number of connected neurons that contributed to LFP is 1, 7 and 11, respectively. The timing step of time delay is 0.03125 ms.}
  \begin{figure}[h]
    \centering
    \subfigure{\includegraphics[width = 0.32\linewidth]{May03/tdmi-tdlc-40-1-003125.png}}
    \subfigure{\includegraphics[width = 0.32\linewidth]{May03/tdmi-tdlc-40-7-003125.png}}
    \subfigure{\includegraphics[width = 0.32\linewidth]{May03/tdmi-tdlc-40-11-003125.png}}
  \end{figure}
  \small{The result would be checked again after debugging.}
\end{frame}

\begin{frame}
  \frametitle{Strange behavior for fewer Noise Ratio}
  \scriptsize{More analysis about the causality of pre-network neuron and post-network neurons would be done.}
  \begin{figure}[h]
    \centering
    \includegraphics[width = 0.72\linewidth]{May03/t5/tdmi-14-1-1000-10000-50-025-60-100.eps}
  \end{figure}
\end{frame}

\begin{frame}
  \frametitle{System with spike transmiting delay}
  \begin{figure}[h]
    \centering
    \subfigure{\includegraphics[width = 0.48\linewidth]{May03/t2/tdmi-0-1-1000-10000-50-025-60-100.eps}}
    \subfigure{\includegraphics[width = 0.48\linewidth]{May03/t5/tdmi-0-1-1000-10000-50-025-60-100.eps}}
  \end{figure}
  \scriptsize{In the left subfigure, 2 ms spike transmiting delay is applied. According to this result, the primary hypothesis is that TDMI analysis can indicate the direction of the network system.}
\end{frame}

\section{To do list}
\begin{frame}
  \frametitle{To do list}
  \begin{enumerate}
    \item Seperate the excitatory current term and inhibitory term in LFP, and analysis the causality of interactions between different types of neurons across networks.
    \item Check the code of calculation of linear correlation.
    \item Return to the previous simulating setups. Check whether TDMI analysis can indicates direction of information flow among all dynamic region or within specific circumstances.
  \end{enumerate}
\end{frame}
% \section{Power Spectrum Analysis for Network dynamics}
%
% \section{Understanding for Paper Reading}

\end{document}
